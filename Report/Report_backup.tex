% Template for ICASSP-2010 paper; to be used with:
%          mlspconf.sty  - ICASSP/ICIP LaTeX style file adapted for MLSP, and
%          IEEEbib.bst - IEEE bibliography style file.
% --------------------------------------------------------------------------
\documentclass{article}
\usepackage{amsmath,graphicx,02460}



\toappear{02460 Advanced Machine Learning, DTU Compute, Spring 2015}


\def\x{{\mathbf x}}
\def\L{{\cal L}}

\title{Relating EEG microstates to the Default Mode Network}
\name{Magnus Alexander Bitsch (s103243), 
Franciszek Olaf Zdyb (s093086),
Anna Maria Walach (s121540) }
\address{Technical University Of Denmark}
%
% For example:
% ------------
%\address{School\\
%	Department\\
%	Address}
%
% Two addresses (uncomment and modify for two-address case).
% ----------------------------------------------------------
%\twoauthors
%  {A. Author-one, B. Author-two\sthanks{Thanks to XYZ agency for funding.}}
%	{School A-B\\
%	Department A-B\\
%	Address A-B}
%  {C. Author-three, D. Author-four\sthanks{The fourth author performed the work
%	while at ...}}
%	{School C-D\\
%	Department C-D\\
%	Address C-D}
%
\begin{document}
%\ninept
%

\maketitle
%
\begin{abstract}
The abstract should appear at the top of the left-hand column of text, about
0.5 inch (12 mm) below the title area and no more than 3.125 inches (80 mm) in
length.  Leave a 0.5 inch (12 mm) space between the end of the abstract and the
beginning of the main text.  The abstract should contain about 100 to 150
words, and should be identical to the abstract text submitted electronically
along with the paper cover sheet.  All manuscripts must be in English, printed
in black ink.
\end{abstract}
%
\begin{keywords}
machine learning, EEG, fMRI, microstates, Default Mode Network
\end{keywords}
%
\section{Introduction}
\label{sec:intro}
\subsection{Microstates}
Microstates are unique topographic distributions of the electrical field 
potential in the brain  \cite{Khanna2015105}. They are transient, patterned and quasi stable 
(\~100 ms). They are derived from EEG signal using either temporal clustering 
or temporal ICA. Microstate analysis has been used for assessing the function of large-scale brain networks.

\subsection{Default Mode Network}
Resting State Networks (RSNs) are networks of brain regions, that are active when a person is resting (but not sleeping). Default Mode Network (DMN) is one of the most researched RSNs. It is becoming active when one mind is "wandering". Its subsystems include part of the medial temporal lobe for memory, part of the medial prefrontal cortex for theory of mind, and the posterior cingulate cortex for integration, along with the adjacent ventral precuneus and the medial, lateral and inferior parietal cortex \cite{dmn_description}.
\subsection{Motivation}
\label{sec:motivation}
Alterations in DMN have been connected to various neurological diseases, like Alzheimer’s or schizophrenia \cite{Yuan20122062}, \cite{Khanna2015105}. DMN can be easily detected by fMRI scanning, but it is an expensive procedure requiring a visit to the hospital. This research aims to find a correlation between microstates derived from EEG data and DMN from fMRI. This may allow the use of EEG as a cheaper and more portable tool for diagnosis of neurological diseases.

\subsection{Problem}


\section{Data}
\label{sec:data}
The EEG data contains 10 minutes probes from 20 subjects, recorded in free different settings: in atmospheric, increased $CO_2$ and increased $O_2$ conditions. The data was recorded in Glostrup Hospital by Egill Rostrup and Ulrich Lindberg as a simultaneous EEG/fMRI. The recording included 30 electrodes for brain activity measurement, one for eye movement and one for heartbeat. The sampling frequency was 500 Hz. The moment of launching the fMRI is recorded for each sample, so the data can be trimmed appropriately.
\subsection{Artefact removal}
Initial cleaning of the data, especially removing the fMRI artefacts, has been performed 
by Glostrup Hospital staff,  followed by further artefact removal performed by Andreas Trier Poulsen. This process included notch- and low-pass filtering, as well as using ECG and EOG to get rid of eye blinks and heart beat artefacts. 
The data from only 5 subjects (in all conditions) remained.

\section{Methods for generating microstates}
% Given artifact-free EEG data from five subjects, we generated 30 microstates using temporal ICA (fig.2).
\subsection{GFP}
Following Yuan et al., we calculated the global field power time course (GFP), and found topographies corresponding to peaks in the GFP. The parameters of the peak detection algorithm were selected for highest correlation with the fMRI data via a grid search.
\subsection{FastICA}
We concatenated the topographies across subjects, and run ICA. Our algorithm of choice was FastICA \cite{fast_robust} with the cubic non-linearity function. The resulting separation matrix was applied to the continuous EEG to obtain time courses of the microstates.
\subsection{Hemodynamic response function}   
We constructed a binary matrix which encodes which microstate has the highest activation at each time, and convolved it with the hemodynamic response function to adjust for the time-delay of the BOLD fMRI response. These were used as regressors in the elastic net model \cite{hastie01}. 
\\ \\
In order to evaluate the robustness of the algorithm of finding microstates, we altered the minimum peak width and height in the peak detection and performed 5-fold cross validation omitting one subject at each fold. In each fold, for a specific set of parameters in the peak detection, we found a set of microstates ordered by the power explained. We aligned the matrices to be of the same sign using correlation of the first column of the mixing matrices with respect to the first fold. We calculated the sum of the Frobenius norm of the deviation from the mean of the mixing matrices. We found that the robustness decreased in both directions. Hence, no restrictions should be given for the peak detection algorithm for the maximum robustness.
\section{Finding relation between microstates and DMN}
\subsection{Independent Components from fMRI}
\subsection{Elastic net}
\subsection{Cross-validation}



\section{Experiments}
\subsection{Adjusting pick sizes}
We performed 5-fold cross-validation with λ log-spaced in the interval 
[ $10^{-4}$ - $10^2$] by randomly dividing the data into the 5 folds to adjust of the fact that the behaviour of a subject might 'change' over time in the scanner 
e.g. fall asleep. Since we did not have an independent test set, we couldn't assess the generalization error. Instead, we estimated the error on the validation set.
\section{Results}
\section{Discussion}
\section{ILLUSTRATIONS, GRAPHS, AND PHOTOGRAPHS}
\label{sec:illust}

Illustrations must appear within the designated margins.  They may span the two
columns.  If possible, position illustrations at the top of columns, rather
than in the middle or at the bottom.  Caption and number every illustration.
All halftone illustrations must be clear black and white prints.  Colors may be
used, but they should be selected so as to be readable when printed on a
black-only printer.

Since there are many ways, often incompatible, of including images (e.g., with
experimental results) in a LaTeX document, below is an example of how to do
this \cite{Lamp86}.

% Below is an example of how to insert images. Delete the ``\vspace'' line,
% uncomment the preceding line ``\centerline...'' and replace ``imageX.ps''
% with a suitable PostScript file name.
% -------------------------------------------------------------------------
\begin{figure}[htb]

\begin{minipage}[b]{1.0\linewidth}
  \centering
%  \centerline{\includegraphics[width=8.5cm]{image1}}
%  \vspace{2.0cm}
  \centerline{(a) Result 1}\medskip
\end{minipage}
%
\begin{minipage}[b]{.48\linewidth}
  \centering
%  \centerline{\includegraphics[width=4.0cm]{image3}}
%  \vspace{1.5cm}
  \centerline{(b) Results 3}\medskip
\end{minipage}
\hfill
\begin{minipage}[b]{0.48\linewidth}
  \centering
%  \centerline{\includegraphics[width=4.0cm]{image4}}
%  \vspace{1.5cm}
  \centerline{(c) Result 4}\medskip
\end{minipage}
%
\caption{Example of placing a figure with experimental results.}
\label{fig:res}
%
\end{figure}

% To start a new column (but not a new page) and help balance the last-page
% column length use \vfill\pagebreak.
% -------------------------------------------------------------------------
\vfill
\pagebreak

\section{Conclusions}
\section{Acknowledgements}
\bibliographystyle{IEEEbib}
\bibliography{refs}

\end{document}
